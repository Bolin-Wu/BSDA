
\subsubsection*{General information}


\begin{itemize}

\item The recommended tool in this course is R (with the IDE R-Studio). You can download R \href{https://cran.r-project.org/}{\textbf{here}} and R-Studio \href{https://www.rstudio.com/products/rstudio/download/}{\textbf{here}}. There are tons of tutorials, videos and introductions to R and R-Studio online. You can find some initial hints from \href{https://education.rstudio.com/}{\textbf{RStudio Education pages}}.
\item When working with R, we recommend writing the report using R markdown and the provided \href{https://github.com/MansMeg/BSDA/blob/main/templates/assignment_template.rmd}{\textbf{R markdown template}}.
The remplate includes the formatting instructions and how to include code and figures.
\item Instead of R markdown, you can use other software to make the PDF report, but the the same instructions for formatting should be used. These instructions are available also in \href{https://github.com/MansMeg/BSDA/blob/main/templates/assignment_template.pdf}{\textbf{the PDF produced from the R markdown template}}.
\item We supply a Google Colab notebook that can also be used for the assignments. We have included installation of all nessecary R packages and hence this can be an alternative to using your own local computer. You can find the notebook \href{https://github.com/MansMeg/BSDA/blob/main/templates/bsda_colab_template.ipynb}{\textbf{here}}. You can also open the notebook in Colab \href{https://colab.research.google.com/github/MansMeg/BSDA/blob/main/templates/bsda_colab_template.ipynb}{\textbf{here}}.
\item Report all results in a single and \emph{anonymous} pdf.
\item The course has its own R package \texttt{bsda} with data and functionality to simplify coding. To install the package just run the following (upgrade="never" skips question about updating other packages):
\begin{enumerate}
\item \texttt{install.packages("remotes")}
\item \texttt{remotes::install\_github("MansMeg/BSDA", \\ subdir = "rpackage", upgrade="never")}
\end{enumerate}
\item Many of the exercises can be checked automatically using the R package \\ \texttt{markmyassignment}. Information on how to install and use the package can be found \href{https://cran.r-project.org/web/packages/markmyassignment/vignettes/markmyassignment.html}{\textbf{here}}. There is no need to include \texttt{markmyassignment} results in the report.
\item Common questions and answers regarding installation and technical problems can be found in \href{https://github.com/MansMeg/BSDA/blob/main/FAQ.md}{Frequently Asked Questions (FAQ)}.
\item Deadlines and information on how to turn in the assignments can be found in Studium.
\item You are allowed to discuss assignments with your friends, but it is not allowed to copy solutions directly from other students or from internet. Try to solve the actual assignment problems with your own code and explanations. Do not share your answers publicly. Do not copy answers from the internet or from previous years. We compare the answers with urkund. All suspected plagiarism will be reported and investigated.
\item If you have any suggestions or improvements to the course material, please post in the course chat feedback channel, create an issue, or submit a pull request to the public repository \href{https://github.com/MansMeg/BSDA/issues}{\textbf{here}}
\end{itemize}
